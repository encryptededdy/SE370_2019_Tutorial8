\documentclass{beamer}
%\documentclass[notes=only]{beamer}

\usepackage[utf8]{inputenc}
\usepackage{listings}
\usepackage{ulem}
\usepackage{hyperref}
\usepackage{amsmath}

\lstdefinestyle{customcbig}{
  belowcaptionskip=1\baselineskip,
  breaklines=true,
%  frame=L,
  xleftmargin=\parindent,
  language=C,
  showstringspaces=false,
  basicstyle=\small\ttfamily,
  keywordstyle=\bfseries\color{green!40!black},
  commentstyle=\itshape\color{purple!40!black},
  identifierstyle=\color{blue},
  stringstyle=\color{orange},
}

\lstdefinestyle{customc}{
  belowcaptionskip=1\baselineskip,
  breaklines=true,
%  frame=L,
  xleftmargin=\parindent,
  language=C,
  showstringspaces=false,
  basicstyle=\scriptsize\ttfamily,
  keywordstyle=\bfseries\color{green!40!black},
  commentstyle=\itshape\color{purple!40!black},
  identifierstyle=\color{blue},
  stringstyle=\color{orange},
}

\lstdefinestyle{customctiny}{
  belowcaptionskip=1\baselineskip,
  breaklines=true,
%  frame=L,
  xleftmargin=\parindent,
  language=C,
  showstringspaces=false,
  basicstyle=\tiny\ttfamily,
  keywordstyle=\bfseries\color{green!40!black},
  commentstyle=\itshape\color{purple!40!black},
  identifierstyle=\color{blue},
  stringstyle=\color{orange},
}
 
\usetheme{Dresden}

\title[SOFTENG 370 Tutorial 9 (2019)] %optional
{Exam Revision}
  
\author{Edward Zhang}
 
% \institute[UoA] % (optional)
% {
%   Department of ECSE\\
%   The University of Auckland
% }
 
\date[October 2019] % (optional)
{SOFTENG 370 T8}

\begin{document}
\frame{\titlepage}
\begin{frame}
  \frametitle{Exam Info}
  Your exam will be short answer, not MCQ. That means the exam from 2012 - 2017 aren't very useful. 2018 had a different lecturer for the first $\frac{1}{4}$ so it's not super helpful either.\\
\end{frame}
\section{Adapted MCQs}
\begin{frame}
  \frametitle{Which of the following is not a necessary component of a monitor?}
  \begin{itemize}
    \item Publicly accessible entry points
    \item<alert@2> A readers/writers lock
    \item A scheduler
    \item A shared resource which is protected by the monitor
  \end{itemize}
  \pause
  \begin{block}{Explanation}
    Reader/Writers lock can enhance performance, but is not required.
  \end{block}
\end{frame}
\begin{frame}[fragile]
  \frametitle{The code below uses a compare and swap function “cas”. What is the code doing?}
  \begin{lstlisting}[style=customc]
    add_to_balance(increase):
      previous_amount = balance
      while (!cas(&balance,
        previous_amount,
        previous_amount + increase)):
      previous_amount = balance
  \end{lstlisting}
  \begin{itemize}
    \item It repeatedly increments balance by increase until balance overflows.
    \item It increments balance by increase using a condition variable.
    \item It safely swaps the values of balance with previous\_amount + balance using a wait-free algorithm.
    \item<alert@2> It safely increments balance by increase using a lock-free algorithm.
  \end{itemize}
\end{frame}
\section{Selected 2018 Exam Questions}
\end{document}