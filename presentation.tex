\documentclass{beamer}
%\documentclass[notes=only]{beamer}

\usepackage[utf8]{inputenc}
\usepackage{listings}
\usepackage{ulem}
\usepackage{hyperref}
\usepackage{amsmath}

\lstdefinestyle{customcbig}{
  belowcaptionskip=1\baselineskip,
  breaklines=true,
%  frame=L,
  xleftmargin=\parindent,
  language=C,
  showstringspaces=false,
  basicstyle=\small\ttfamily,
  keywordstyle=\bfseries\color{green!40!black},
  commentstyle=\itshape\color{purple!40!black},
  identifierstyle=\color{blue},
  stringstyle=\color{orange},
}

\lstdefinestyle{customc}{
  belowcaptionskip=1\baselineskip,
  breaklines=true,
%  frame=L,
  xleftmargin=\parindent,
  language=C,
  showstringspaces=false,
  basicstyle=\scriptsize\ttfamily,
  keywordstyle=\bfseries\color{green!40!black},
  commentstyle=\itshape\color{purple!40!black},
  identifierstyle=\color{blue},
  stringstyle=\color{orange},
}

\lstdefinestyle{customctiny}{
  belowcaptionskip=1\baselineskip,
  breaklines=true,
%  frame=L,
  xleftmargin=\parindent,
  language=C,
  showstringspaces=false,
  basicstyle=\tiny\ttfamily,
  keywordstyle=\bfseries\color{green!40!black},
  commentstyle=\itshape\color{purple!40!black},
  identifierstyle=\color{blue},
  stringstyle=\color{orange},
}
 
\usetheme{Dresden}

\title[SOFTENG 370 Tutorial 9 (2019)] %optional
{Exam Revision}
  
\author{Edward Zhang}
 
% \institute[UoA] % (optional)
% {
%   Department of ECSE\\
%   The University of Auckland
% }
 
\date[October 2019] % (optional)
{SOFTENG 370 T8}

\begin{document}
\frame{\titlepage}
\begin{frame}
  \frametitle{Exam Info}
  Your exam will be short answer, not MCQ. That means the exam from 2012 - 2017 aren't very useful. 2018 had a different lecturer for the first $\frac{1}{4}$ so it's not super helpful either.\\
\end{frame}
\section{Adapted MCQs}
\begin{frame}
  \frametitle{Which of the following is not a necessary component of a monitor?}
  \begin{itemize}
    \item Publicly accessible entry points
    \item<alert@2> A readers/writers lock
    \item A scheduler
    \item A shared resource which is protected by the monitor
  \end{itemize}
  \pause
  \begin{block}{Explanation}
    Reader/Writers lock can enhance performance, but is not required.
  \end{block}
\end{frame}
\begin{frame}
  \frametitle{Which of the following best explains what happens when a damaged C program comes to an end but doesn’t call the exit routine?}
  \begin{itemize}
    \item The damaged program can corrupt memory used by other processes and cause them to crash or perform illegal instructions.
    \item<alert@2> The operating system takes control when the program tries to execute an illegal instruction or attempts to access unallocated memory.
    \item The C standard library takes control when the program fails to return to the code which called the main function.
    \item The operating system creates a new process and restarts the damaged program in that process so that it gets another chance to complete.
  \end{itemize}
\end{frame}
\begin{frame}[fragile]
  \frametitle{The code below uses a compare and swap function “cas”. What is the code doing?}
  \begin{lstlisting}[style=customc]
    add_to_balance(increase):
      previous_amount = balance
      while (!cas(&balance,
        previous_amount,
        previous_amount + increase)):
      previous_amount = balance
  \end{lstlisting}
  \begin{itemize}
    \item It repeatedly increments balance by increase until balance overflows.
    \item It increments balance by increase using a condition variable.
    \item It safely swaps the values of balance with previous\_amount + balance using a wait-free algorithm.
    \item<alert@2> It safely increments balance by increase using a lock-free algorithm.
  \end{itemize}
\end{frame}
\begin{frame}
  \frametitle{Which of the following does NOT happen in a context switch between threads in the same process?}
  \begin{itemize}
    \item The processor registers for the currently running thread are saved.
    \item The processor registers are loaded with the saved values for the new thread.
    \item<alert@2> The page table is switched from the old thread to the new thread.
    \item The thread states for the two threads may be changed.
    \item The stack is changed from the old thread to the new thread.  
  \end{itemize}
  \pause
  \begin{block}{Explanation}
    Memory is shared between threads, so same page table.
  \end{block}
\end{frame}
\begin{frame}
  \frametitle{Which of the following is False?}
  \begin{itemize}
    \item FUSE works by redirecting file operations through the FUSE module to a process running in user mode.
    \item To use a FUSE file system we mount the file system over an existing directory.
    \item<alert@2> To use FUSE requires root privileges.
    \item If the FUSE process is killed the files and directories contained within it will not be accessible.
    \item There has to be a FUSE kernel module in order for FUSE to work on Linux.
  \end{itemize}
  \pause
  \begin{block}{Explanation}
    You probably used FUSE w/o root in your assignment.
  \end{block}
\end{frame}
\begin{frame}
  \frametitle{Which of the following disk scheduling algorithms are commonly used for scheduling SSDs?}
  \begin{itemize}
    \item Shortest Seek time First
    \item SCAN
    \item<alert@2> First come, first served
    \item Circular SCAN
    \item None of the above
  \end{itemize}
  \pause
  \begin{block}{Explanation}
    SSDs have no Seek time, and no head/platter so SCAN is irrelevant. FCFS makes sense since no special handling is required.
  \end{block}
\end{frame}
\begin{frame}
  \frametitle{What causes thrashing?}
  \begin{itemize}
    \item When the foreground process has completely used up the number of frames it has been allocated.
    \item<alert@2> When the sum of the pages of the working-sets exceeds the number of frames.
    \item When there is not enough contiguous memory to be allocated for all current working sets.
    \item When all frames are currently being used.
    \item When all processes have filled up their page tables.
  \end{itemize}
  \pause
  \begin{block}{Explanation}
    Recall that thrashing is when the virtual memory system is overused, and is thus stuck in a constant state of paging / pagefaults.
  \end{block}
\end{frame}
\begin{frame}
  \frametitle{Which of the following statements about user level device drivers is FALSE?}
  \begin{itemize}
    \item<alert@2> User level drivers cannot deal with device interrupts.
    \item User level drivers can communicate with memory mapped devices.
    \item Most problems with user level drivers do not affect the kernel.
    \item Because of mode transitions user level drivers are sometimes not used for fast devices.
    \item User level drivers can communicate with IO ports
  \end{itemize}
\end{frame}
\begin{frame}
  \frametitle{Which of the following conditions have a direct influence on the effective access time of memory?}
  \begin{itemize}
    \item The time to retrieve page table information for the page.
    \item The page fault rate.
    \item The time taken to provide a free frame for the page if required.
    \item<alert@2> All of the above.
  \end{itemize}
  \pause
  \begin{block}{Explanation}
    Recall that effective means we take into account the average time.
  \end{block}
\end{frame}
\begin{frame}
  \frametitle{Which of the following best explains what happens concerning the current instruction when a page fault occurs?}
  \begin{itemize}
    \item The instruction is completed and then the required page is brought into a frame so that the program can continue correctly.
    \item The current thread changes to another thread and the offending instruction and its thread are removed from the system.
    \item<alert@2> After the required page has been brought into a frame the instruction must usually be restarted from its initial state before the page fault occurred.
    \item The current instruction is only restarted if it caused the page fault. If it didn’t cause the page fault it continues normally.
  \end{itemize}
\end{frame}

\section{Selected 2018 Exam Questions}
\begin{frame}
  \frametitle{Question 9b}
  One protection against the Meltdown exploit that has been implemented in operating systems is kernel page-table isolation (KPTI). Explain what kernel page-table isolation is.
  \pause
  \begin{block}{Answer}
    Separate page tables are kept for a process when it is running in kernel and user mode. The user mode page tables do not have most of the kernel pages mapped into them.
  \end{block}
\end{frame}
\begin{frame}
  \frametitle{Question 9c/d}
  What effect could KPTI have on efficiency, and how does it prevent Meltdown?
  \pause
  \begin{block}{Efficiency}
    Anything that switches to kernel mode, like syscalls, requires a new page table to be loaded. This will also likely flush the TLB.
  \end{block}
  \pause
  \begin{block}{Preventing Meltdown}
    Meltdown's timing attack relies on attempts to access values in kernel address space (even if these are eventually rejected by a privledge check). However, with KPTI, those addresses are not accessible in the current address space, so the attempt cannot be made.
  \end{block}
\end{frame}
\begin{frame}
  \frametitle{Question 10}
  What is a user-level device driver?
  \pause
  \begin{block}{Answer}
    A device driver implemented in code to be run in user mode rather than kernel mode.
  \end{block}
\end{frame}
\begin{frame}
  \frametitle{Question 4c}
  Explain what a Least Slack Time schedule is and why it is a suitable choice for real-time scheduling.
  \pause
  \begin{block}{Answer}
    The slack time is the amount of time from the current time to the end of the deadline for a process minus the amount of compute time remaining for that process. A Least Slack Time schedule chooses the process with the smallest slack time at any point to be the scheduled process. This is appropriate for a real-time system because processes have to complete before their deadlines or the real-time requirement has not been met. When the slack time of a process gets to zero it has to run or it will pass its deadline.
  \end{block}
\end{frame}
\section{Selected 2011 Exam Questions}
\begin{frame}
  \frametitle{Question 9b}
  One protection against the Meltdown exploit that has been implemented in operating systems is kernel page-table isolation (KPTI). Explain what kernel page-table isolation is.
  \pause
  \begin{block}{Answer}
    Separate page tables are kept for a process when it is running in kernel and user mode. The user mode page tables do not have most of the kernel pages mapped into them.
  \end{block}
\end{frame}
\begin{frame}
  \frametitle{Question 2b/c}
  Give an example of when a hard link is preferable to a soft link
  \pause
  \begin{block}{Answer}
    When we want to avoid the problem of a broken link when the original file gets moved.
  \end{block}
  \pause
  Give an example of when a soft link is preferable to a hard link
  \pause
  \begin{block}{Answer}
    When the link is to a file on a different device or volume, or to a directory.
  \end{block}
\end{frame}
\begin{frame}
  \frametitle{Question 4a}
  Using virtual memory slows down the effective access time, EAT, of memory for our processes. In lectures we saw two different ways in which access to a particular address in  memory can be slowed down compared to a non-virtual memory system. Describe the two different ways and relate both of them to the page table information.
  \begin{block}{Hint}
    Consider what extra steps are added when using virtual memory.
  \end{block}
\end{frame}
\begin{frame}
  \frametitle{Question 4a Answer}
  1. The page table information needs to be accessed forevery memory access. This either leads to two (or more)memory reads for every memory access or else requires acache (the TLB) to hold the page table information forrecently accessed pages. \\
  2. The page may not be resident in RAM but on disk. Inorder to access data on the page it has to be brought intoa frame of real memory. The information about whether thepage is in RAM or on disk is stored in the page tableentry.
\end{frame}
\end{document}